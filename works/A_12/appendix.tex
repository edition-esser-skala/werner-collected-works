% latexmk -cd -lualatex works/A_12/appendix.tex
% latexmk -c -cd works/A_12/appendix.tex

\documentclass{ees}

\newlist{lyricslist}{description}{1}
\setlist[lyricslist]{
  partopsep=0pt,
  labelindent=0em,
  labelwidth=4.5em,
  leftmargin=4.5em,
  labelsep=0pt,
  first=\ltseries,
  font=\normalfont\itshape\ltseries,
  style=multiline
}

\newenvironment{lyrics}[1]{%
  \subsection{#1}\nopagebreak%
  \begin{lyricslist}%
  \let\voice\item%
}{%
  \end{lyricslist}%
}

\begin{document}

\pagenumbering{arabic}
\setcounter{page}{1}

\chapter{Lyrics}

\section{Actus primus}

\begin{lyrics}{Scena prima}
  \voice[Job]
  Ô mich beglückhter Job!\\
  wie werd ich wohl bezahlen\\
  mein höchſten Gott und Herr\\
  die überhäuffte Gnaden?\\
  mit den Er mich vor allen\\
  pflegt gnädigſt zu beladen.\\
  Ô! ô! daß ich doch mit Lob\\
  ihm gnugſam könt beflen;\\
  und mich mit ſchöngeſtalten Kindern,\\
  mit Schaff, Camel und fetten Ründern\\
  ſo groß geſegnet hat.

  \voice[Eliphas]
  Gar wohl, mein werther Freund!\\
  der du vor ſolcher Gnad\\
  dem Schöpffer billich danckhest,\\
  hievon niemahlen wanckheſt,\\
  weill ers ſo gut vermeint.\\
  Dein Opfer, ſo du täglich\\
  ihm pflegeſt abzulegen,\\
  wird dir noch größten Segen\\
  von Gott erbittend machen.

  \voice[Job]
  Wollan\\
  mein Hertz ſodan\\
  ſoll ſtäts zu ihme wachen.
\end{lyrics}

\begin{lyrics}{Aria prima}
  \voice[Job]
  Großer Herſcher, deine Gnade,\\
  die du pflegeſt außzutheilen,\\
  ſein zuweillen nur zum Schade\\
  die zur Wolluſt gneiget ſein.\\[1ex]
  Ich ſuech bloß dich anzuflehen,\\
  dein Gebott recht zu erfüllen,\\
  nur dein Wille ſoll geſchehen,\\
  ſo verbleibt mein Gwißen rein.
\end{lyrics}

\begin{lyrics}{Scena secunda}
  \voice[Jobs Frau]
  Beglückht biſtu mein Job!\\
  Und wirſt auch ſtets geſegnet ſein\\
  ſambt mir und allen Kündn dein,\\
  ſolang als Gottes Lob\\
  in deinen Mund erſchallet.

  \voice[Job]
  Mit allen dem iſt lang noch nicht\\
  nach Menſchenpflicht\\
  die gringſte Schuld bezahlet.\\
  Doch weil ſich Gott begnügen laßt\\
  mit unſrer Wenigkeit,\\
  ſo ſey demnach der Schluß gefaßt,\\
  daß ich zu jederzeit\\
  ihm täglich Opfer reiche,\\
  damit hinführ von mir und dir\\
  all Schad und Uebel weiche.

  \voice[Jobs Frau]
  Der Vorſatz iſt gemacht,\\
  nun iſt mein Bitt\\
  daß dißes Glübt\\
  von dir auch werd volbracht!
\end{lyrics}

\begin{lyrics}{Aria secunda}
  \voice[Jobs Frau]
  Wer Gott will was angeloben,\\
  und das Werckh wird auffgeſchoben,\\
  diſer ſeye wohl vergwißt,\\
  daß er mit dergleichen Hertzen\\
  gar nicht pflege vill zu ſchertzen\\
  wan man auch ſein Glübt vergißt.\\[1ex]
  Gott iſt von Natur beſchaffen,\\
  jene Heychler abzuſtraffen,\\
  die da nur auf bloßen Schein\\
  in der Noth zwahr vill verſprechen,\\
  dannoch ſolches unterbrechen\\
  wan ſie in den Wohlſtand ſeyn.
\end{lyrics}

\begin{lyrics}{Scena tertia}
  \voice[Die Kinder Gottes]
  Großer Gott und König!\\
  Du Herrſcher aller Weld,\\
  vor dir ſich alles unterthänig\\
  zu deinen Dienſten ſtellt.

  \voice[Schöpffer]
  Sathan! Sage an?\\
  Wo kommeſtu hieher\\
  und was iſt dein Verlangen?

  \voice[Leviathan]
  Ich bin, mein Gott und Herr,\\
  in Land herum gegangen.

  \voice[Schöpffer]
  Haſtu auch wohl betrachtet\\
  Job, meinen frommen Knecht?\\
  als der da ſchlecht und grecht\\
  in ſeiner Unſchuld lebet,\\
  nur nach den gueten ſtrebet,\\
  und ſeinen Gott hochachtet.

  \voice[Leviathan]
  Vermeinſtu dan, daß Job\\
  umſonst dich alſo ehret,\\
  haſtu nicht all ſein Guet,\\
  ſein Hauß und Viech vermehret.\\
  Wie ſoll dich dan ein Menſch\\
  nicht billich lob- und preyſen,\\
  dem du ſo große Schätz\\
  und Gnaden thuſt erweiſen.\\
  Streckh nur dein ſtarkhe Hand\\
  an ihm ein wenig auſ,\\
  entnehme ſeine Güetter,\\
  ſpolir das gantze Hauß,\\
  hiemit kanſtu verſuchen,\\
  ob Er nicht ſein Gebiether\\
  trutz einem in den Land\\
  ins Angeſicht wird fluechen.

  \voice[Schöpffer]
  Wollan! ſo gehe dan,\\
  ſieh alles was er hat\\
  durch mein beſondre Gnad\\
  ſey dir nun freygeſtellet,\\
  mach wie es dir gefället,\\
  allein an ſeinen Leib\\
  leg deine Hand nicht an!
\end{lyrics}

\begin{lyrics}{Aria tertia}
  \voice[Schöpffer]
  Meine Urtheill ſein verborgen,\\
  die ich niemand kündig mach,\\
  aller Menſchen Witz und Sorgen\\
  ſein hierinfahls vill zu ſchwach.\\[1ex]
  Ich pfleg ſolche offt zu züchten,\\
  die mirs außerwählet ſeyn,\\
  böſe werd ich einſtens richten\\
  mit der Straff und Höllenpeyn.
\end{lyrics}
\clearpage
\begin{lyrics}{Scena quarta}
  \voice[Leviathan]
  Weil mir dan Gwald gegeben,\\
  meine Hand an Job zu ſtreckhen,\\
  werd ich ihn an allen Eckhen\\
  mörderlich zu quellen trachten,\\
  daß er ſolt vor Ängſten beben,\\
  mithin ſeinen Gott verachten.

  \voice[Job]
  Nun mein großer Gott und Herr,\\
  nehm es dan zu deiner Ehr\\
  diſes Opffer an in Gnade,\\
  ſchütze mich vor allen Schade,\\
  mir dein Huld und Gnad beſcher!

  \voice[Leviathan]
  Hör mein Job, was ich dir melde,\\
  all dein Viech ſo auf dem Felde\\
  hat der Feund hinweg getriben,\\
  all die Knechte auffgeriben\\
  und mit ſeinen Schwerd erſchlagen,\\
  ich allein bin noch entronnen\\
  und ſo vill der Zeit gewunnen,\\
  daß ich dir könt ſolches klagen.

  \voice[Job]
  Gott hats geben und genohmen,\\
  alles iſt durch ihn gekhommen,\\
  wies dem Herrn gefallen hat\\
  also iſt es auch geſchehen,\\
  dannoch werd ich frueh und ſpath\\
  ihn umb ſeine Hilff anflehen.

  \voice[Leviathan]
  Ach! waß jammervolle Klagen,\\
  höre, was vor ſchwäre Plagen\\
  über dich verhänget ſeyn:\\
  Von dem Himmel kam daß Feur\\
  auf die Erd herab gefallen,\\
  Menſch und Viech mußt ungeheur\\
  alles mit der Haut bezahlen;\\
  ich bin einzig nur allein\\
  diſem Unheyl noch entrunnen,\\
  auch Gelegenheit gewunnen,\\
  umb dir ſolches noch beyzeiten\\
  in der Wahrheit anzudeithen.

  \voice[Job]\enlargethispage\baselineskip
  Bloß kam ich auß einen Weib,\\
  bloß werd ich zur Erde gehen,\\
  gwiß iſt, daß in meinen Leib\\
  einſtens werde Gott anſehen,\\
  darum ſey zu jederzeit\\
  Gottes Nahm gebenedeyt.

  \voice[Leviathan]
  Job merckh auf, vernehme mich,\\
  es betrifft dein Weib und dich:\\
  als nun deine liebe Kinder\\
  fröhlich waren, und nichts münder\\
  dachten auf ein Todtgefahr,\\
  kam ein Sturmwind alſo ſcharff,\\
  daß ers Hauß zu Boden warff,\\
  alles iſt zugrund gerichtet,\\
  ich allein hab mich geflüchtet,\\
  dir die Poſt zu hinterbringen,\\
  daß ſie mit dem Tode ringen,\\
  nun iſt all dein Hoffnung gar.

  \voice[Job]
  Ey, ſo lebet doch mein Schöpffer,\\
  jener große Menſchentöpffer,\\
  der auß Laim mich hat formirt,\\
  dieſen werd ich dannoch preyſen,\\
  alles Lob und Ehr erweiſen,\\
  ob Er ſchon ſein Gfäß probirt.

  \voice[Leviathan]
  Ich merckh ſchon, auf ſolche Weiß\\
  kom ich nicht zu Ehr und Preiß,\\
  es muß weißlicher geſchehen,\\
  ſonſt werd ich den Krebßgang gehen.
\end{lyrics}

\begin{lyrics}{Aria quarta}
  \voice[Leviathan]
  Mein Verſuchung iſt vergeben,\\
  alß ſo lang der Menſch im Leben\\
  Gottes Gnad bey ſich verſpürth,\\
  einen fromen Wandel fürth.\\[1ex]
  Iſt er hievon abgewichen,\\
  komm ich leichtlich eingeſchlichen\\
  und zur Boßheit ihn verleith,\\
  ſo er (dan) ewig nachbereut.
\end{lyrics}

\begin{lyrics}{Scena quinta}
  \voice[Die Kinder Gottes]
  Großer Gott und König!\\
  Du Herrſcher aller Weld,\\
  vor dir ſich alles unterthänig\\
  zu deinen Dienſten ſtellt.

  \voice[Schöpffer]
  Sathan, ſage an, wo kommeſtu hieher\\
  und waß iſt dein Verlangen?

  \voice[Leviathan]
  Ich bin, mein Gott und Herr,\\
  in Land herum gegangen.

  \voice[Schöpffer]
  Haſtu auch wohl betrachtet\\
  Job, meinen frommen Knecht,\\
  als der da ſchlecht und grecht\\
  in ſeiner Unſchuld lebet,\\
  nur nach dem Gutten ſtrebet\\
  und ſeinen Gott hochachtet.\\
  Du aber haſt mein Hertz beweget,\\
  daß ich ihn mit ſwchären Plagen\\
  ſambt ſeinen ganczen Hauß beleget,\\
  und müßt ihm Leuth und Viech erſchlagen.

  \voice[Leviathan]
  Haut umb Haut und waß der Menſch beſitzet,\\
  diß laßt er vor ſein Leben,\\
  mithin iſt er beyneben\\
  ſchon ſicher und vertraut,\\
  wann nur der Leib beſchützet.\\
  Allein, ſtreckh deine Hand\\
  bey Job noch fehrner an,\\
  mit Schmertzen ihn verſuche,\\
  ob er in ſolchen Stand\\
  dich als ein bherzter Mann\\
  ins Angeſicht nicht flueche.

  \voice[Schöpffer]
  Auch diß will dir erlauben,\\
  allein an ſeinen Leben\\
  iſt dir kein Macht gegeben,\\
  deß ſolſt ihn nicht berauben.

  \voice[Leviathan]
  Nun werd ich mich erſt laben,\\
  er ſolle bald ein ander Gſtald,\\
  dein frommer Diener, haben.

  \voice[Job]
  Ach! wie ſchwär werd ich gepreßet,\\
  daß ich doch zu diſer Stund\\
  alſo gleich nur ſterben kunt!\\
  Seht, wie doch von Haubt zum Füßen\\
  alle Glider leyden müßen,\\
  wie das Eyter herumfreßet.\\
  Meine Worth ſein voll der Schmertzen,\\
  Seuffzer ſteigen auß dem Hertzen,\\
  wilſtu dan, ô Menſchenhüetter,\\
  dein Geſchöpff ſogar verderben?\\
  Ey, du großer Weldgebiether,\\
  laß mich doch des Todes ſterben!
\end{lyrics}
\clearpage
\begin{lyrics}{Scena sexta}
  \voice[Jobs Frau]
  Ey, ey, wie gar ein frommen Mann\\
  habe ich doch überkhommen,\\
  jezo ſehe ich den Lohn,\\
  wie das Glickh hat abgenohmen.\\
  Kennſtu deine Einfald nicht,\\
  wilſtu fehrner dich noch härben?\\
  Seegne Gott nach deiner Pflicht,\\
  dann du wirſt in Kürtze ſterben.

  \voice[Job]
  Du redeſt als ein törricht Weib\\
  und achteſt nicht der Sünden.\\
  Ach mögſtu nur an deinen Leib\\
  der tauſende empfünden!\\
  Ich liege hier gleich einem Viech\\
  und weltze mich mit Koth und Wuſt umbgeben;\\
  ô wohl ein Jammerleben!\\
  dergleichen nie gefunden.\\
  Ach! daß ich nur bald\\
  in bleicher Todsgeſtald\\
  deß Schmertzens wurd entbunden!
\end{lyrics}

\begin{lyrics}{Scena septima}
  \voice[Eliphas]
  Die Peyn iſt alzu groß,\\
  hier muß man billich ſchweigen,\\
  es kan ſich wohl daß Loß\\
  auf unſern Ruckhen zeigen.

  \voice[Jobs Frau]
  Allein er greifft den Schöpffer an.

  \voice[Eliphas]
  Diß kan ich ſchwärlich glauben.

  \voice[Job]
  Ach thut mir doch erlauben!

  \voice[Jobs Frau]
  Er iſt dem Heüchlen zuegethan.

  \voice[Leviathan]
  Nun hab ich meine Freud daran.

  \voice[Schöpffer]
  Und du wirſt nicht obſigen.

  \voice[Job]
  Doch muß ich unterligen.\\
  Soll dan ein flüchtig düres Blat\\
  von Wind und Lufft getriben,\\
  ſo gar ohn alle Huld und Gnad\\
  ſein gänzlich aufgerieben?\\
  Siechſt du dan auch mit Menſchenaugen,\\
  die meiſtens nur zum Böſen taugen,\\
  ſeynd deine Jahr auch Menſchenjahr,\\
  daß du nach meiner Sünde ſucheſt,\\
  mich deiner Hände Werckh verflucheſt,\\
  in deme ja vor dir kein Haar\\
  noch Pünctlein mag verborgen ſein.\\
  Du weißt, daß ich nicht gottlos bin,\\
  und würfſt mich doch zur Folterpeyn\\
  auf ein verachtes Beth dahin,\\
  da doch niemand auß deiner Hand\\
  ſich keineswegs erretten kann.\\
  Ô daß ich doch zu diſer Stund\\
  in Abgrund mich verbergen kunt,\\
  ſo wär ich ein beglückhter Mann.

  \voice[Eliphas]
  Mein Freund, du redeſt unbedacht,\\
  wie kan ein Menſch von Gott gemacht\\
  gerecht vor ihn ſich nennen,\\
  muſtu nicht ſelbſt bekhennen,\\
  es waren ja die Engel ſein\\
  nicht alle von der Boßheit rein,\\
  und du wilſt dich beſchönen.
\end{lyrics}

\begin{lyrics}{Aria quinta}
  \voice[Eliphas]
  Alſo ſeyn der Menſchen Gmüth,\\
  offt der Frommen auch ſogar,\\
  daß ſie murren, widerkhurren\\
  gegen jenen Weldgebüether\\
  in der gringſten Leibsgefahr.\\[1ex]
  Solche Kläger ſollen wißen,\\
  daß Gott nur ein kleine Weyl\\
  ſie probire, exercire,\\
  pur zu ihren Seelenheyl.
\end{lyrics}

\begin{lyrics}{[Scena sine numero]}
  \voice[Job]
  Ey laßet mich dan raſten\\
  auf diſen Krankhenbeth,\\
  ihr pflegt nur anzutaſten\\
  mein Gmüth und Hertz ſo voller Schmertz,\\
  gleich denen erzverhaßten.
\end{lyrics}

\begin{lyrics}{Chorus deren Kindern Gottes}
  \voice[Die Kinder Gottes]\enlargethispage\baselineskip
  Seht! ſeht! ſo pfleget Gott zu ſtihlen,\\
  dan nach ſeinen Worth und Willen\\
  wird diß Rund der Weld regirt.\\
  Alles muß ſich unterwerffen,\\
  niemand darff die Zungen ſchärffen,\\
  ihm allein das Recht gebürth.
\end{lyrics}

\section{Pars secunda}

\begin{lyrics}{Scena prima}
  \voice[Jobs Frau]
  Ô daß große Hertzenleyd
  ſo meine Seel empfündet,
  wie, hab ich mich dan villeicht
  geg’n Gott ſo ſchwär verſündet?
  Daß all Hoffnung von mir weicht
  und ſich häufft die Bitterkheit,
  niemand kan den Schmertz errathen,
  ſo mir all mein Mann durchdringt,
  ich leb in den Todtesſchatten,
  der mich in die Gruebe bringt.
  Wan ich mich nun recht beſchau,
  wer ich bin und vor geweſen,
  nemblich ein beglückhte Frau,
  werd ich an der Stirne leſen,
  daß ich ſeye voll der Noth
  und mithin der Menſchen Spott,
  diß macht vor den Jahren grau.

  \voice[Job]
  Ey, bin ich dan auf allen Seithen
  voll der Angſt und Bitterkheit,
  will ſich dan auch der Schmertz außbreithen
  in die lange Ewigkeit,
  warum bin ich nicht umbkhommen,
  da ich gieng auß Mutters Schooß,
  und alſo hinweg genohmen
  wär ich alles Jammers loß.

  \voice[Eliphas]
  Sag, wo iſt nun dein Gedult,
  wodrin all dein guttes Weeſen?
  Haſtu dan niemahl gehört,
  oder irgendwo geleſen,
  daß ein Menſch gantz unverſchuld
  ſey ſogar von Gott verſtoßen,
  auß der Huldſchaft außgeſchloßen,
  warum biſt dir ſelbſt beſchwärt?

  \voice[Job]
  Waß will dan mein Stärckh außweiſen,
  der ich willig leyden ſolt,
  bin ich dan von Stein und Eyſen,
  daß man nich zermallen wolt.
  Ô deß Jamers, wer kan glauben
  diſe große Höllenpeyn,
  muß ich dan gefoltert ſeyn?
  Wan mein Gott mir thät erlauben,
  gieng ich in daß khüele Grab,
  alda könt ich wohl geneſen,
  wäre gleich als nie geweſen,
  und nehm all mein Schmertzen ab.
\end{lyrics}

\begin{lyrics}{Aria sexta · Siciliana}
  \voice[Job]
  Leichtlich iſt geduldig ſeyn,
  wo kein Schmertzen
  in den Hertzen,
  da kein Jammer, (noch) Creütz und Pein.
  Diß iſt ein beherzter Man,
  der nicht klaget,
  noch verzaget,
  in die Noth ſich ſchickhen kan.
\end{lyrics}

% \begin{lyrics}{}
%   \voice[]
% \end{lyrics}


\end{document}
